\documentclass[letterpaper]{article}
\usepackage[english]{babel}
\usepackage[utf8]{inputenc}
\usepackage[top=1.2in, left=0.7in, bottom=1.2in, right=0.7in]{geometry} % sets the margins
\usepackage{amsmath,amsthm,amssymb, amsmath}
\usepackage{url} % fixes url problem
\usepackage{csquotes}% Recommended
\usepackage[doublespacing]{setspace} % turns on double spacing
\usepackage[style=authoryear-ibid,backend=biber]{biblatex}
\addbibresource{references.bib}% Syntax for version >= 1.2
\setlength{\parindent}{0pt}

\title{Estimation of Covariance Matrix}
\author{Refael Lasry}
\date{February 2017}

\begin{document}
	\maketitle
	\tableofcontents
	\include{chap1}
	\include{chap2}

	
	\section{Introduction}
	\subsection{Covariance Matrix}
	Let \(X\) be a \(p \times 1\)  random vector, where each entry \(X_1, X_2, ..., X_p\) is a random variable.
	
	\begin{align*}
		X &= \begin{bmatrix}
			X_{1} \\
			X_{2} \\
			\vdots \\
			X_{j} \\
			\vdots \\
			X_{p}
		\end{bmatrix}
	\end{align*}
	
	The \(p \times p\) covariance matrix \(\Sigma \) defined as follow, \\
	\begin{align*} 
		\Sigma = Cov(X) := E[[X-E(X)][X-E(X)]^T] = 
		\begin{bmatrix}
			(X_1-E(X_1))^2 & ... &  (X_1-E(X_1))(X_p - E(X_p))  \\
			(X_2-E(X_2))(X_1 - E(X_1))  & ...     & (X_2-E(X_2))(X_p - E(X_p))	\\
			\vdots          & \ddots  & \vdots		\\
			(X_p-E(X_p))(X_1 - E(X_1)  & ...     & (X_p-E(X_p))^2
		\end{bmatrix}
	\end{align*}
	
	it can be seen that, 


	\[\Sigma = Cov(X) = E[[X-E(X)][X-E(X)]^T] = E[[X-E[X]][X^T - E[X]^T]] \]  
	\[= E[XX^T - XE(X)^T -E(X)X^T + E(X)E(X)^T] \]  
	\[= E[XX^T] -E[XE(X)^T] -E[E(X)X^T] + E[E(X)E(X)^T] \] 
	\[= E(XX^T) -E(X)E(X)^T -E(X)E(X)^T + E[X]E[X]^T \] 
	\[= E[XX^{T}] -E[X]E[X]^T  \]



	 The covariance matrix \(\Sigma\) sometimes called the population covariance matrix and the estimator \(\hat{\Sigma} \) sometimes called sample covariance matrix. 
	 Using the phrase "population" comes to emphasis that \(\Sigma\)  is containing the variance and the covariance of the "real world", while \(S\) is based on samples from the random vector \(X\).
	 
	\subsection{MLE Estimator}

	Given \(n\) samples from the random vector \(X\),
	\begin{center}
	$X_{1} = \begin{bmatrix}x_{11}  \\x_{12}\\ \vdots\\ x_{1p}  \end{bmatrix}$,
	$X_{2} = \begin{bmatrix}x_{21}  \\x_{22}\\ \vdots\\ x_{2p}  \end{bmatrix}$
	, ... ,
	$X_{n} = \begin{bmatrix}x_{n1}  \\x_{n2}\\ \vdots\\ x_{np}  \end{bmatrix}$
	\end{center}
	each entry \(x_{ij}\) is representing the sample \(i\) of the  parameter \(j\). \\
	
	Putting all the vectors together we get the sample matrix \(X\). 
	
	\begin{align*} 
	X = 
	\begin{bmatrix}
	 \mid & \mid &   & \mid \\
	X_{1} & X_{2} & \cdots & X_{n} \\
	 \mid & \mid & & \mid
	\end{bmatrix}
	=
	\begin{bmatrix}
	x_{11} & x_{21} & \dots     & x_{n1} \\
	x_{12} & x_{22} & \dots     & x_{n2} \\
	\vdots  & \vdots  & \ddots  & \vdots  \\
	x_{1p} & x_{2p} & \dots     & x_{np}
	\end{bmatrix}
	\end{align*}
	
	The MLE unbiased estimator for the covariance matrix \(\Sigma\), denote by \(S\) is,  
	\[S = \frac{1}{N-1}X(I - \frac{1}{N}\boldsymbol{1}
	\boldsymbol{1}')X' \]   
	where \(\textbf{1} \)  is \(N \times 1 \) vector,
	$\boldsymbol{1} = \begin{bmatrix}1  \\1\\ \vdots\\ 1  \end{bmatrix}$
	and \(I\) is \(p \times p\) identity matrix.\\
	
	Number Of Parameters To Estimate including the diagonal is \(\frac{p(p + 1)}{2} \).
	
	
	\subsection{Centering The Sample Matrix}	 
	Removing the average happens all the time while computing the estimator for the covariance
	matrix. To make life easier, from now and on, the sample matrix \(X\) assumed to be centered. From each variable the average of the parameter is removed, \(x_{ij}\)  will become \(x_{ij} -  \frac{1}{n} \sum_{i=1}^{n}x_{ij} \)
	
	
	\begin{align*} 
	X =
	\begin{bmatrix}
	x_{11} - \bar{x}_{\cdot 1} & x_{21} - \bar{x}_{\cdot 1} & \dots     & x_{n1} -\bar{x}_{\cdot n} \\
	x_{12} - \bar{x}_{\cdot 2} & x_{22} - \bar{x}_{\cdot 2} & \dots     & x_{n2} -\bar{x}_{\cdot p} \\
	\vdots  & \vdots  & \ddots  & \vdots  \\
	x_{1p} - \bar{x}_{\cdot p} & x_{2p} - \bar{x}_{\cdot p} & \dots     & x_{np} -\bar{x}_{\cdot p}
	\end{bmatrix}
	\end{align*}
	
	The MLE estimator will get the form of, \( S = E[XX^{T}] - 0 = E[XX^{T}] \) .  
	
	\subsection{Estimation while \(n < p\)} 
	An important case of the estimation is where the number of samples is smaller then the number of the parameters \((n < p)\). The  MLE estimator has bad properties and a solution is needed. Some of the solutions are tapering, regularizations, linear and non-linear shrinkage .The following section will review the solution of linear shrinkage. 

	
	\section{Shrinkage Estimation}
	First method to discuss is shrinkage, it will review through the article \textit{\cite{ledoit_wolf_2003}}. Shrinkage method is combining two estimators unbiased and bias with a proper weight. The bias estimator called shrinkage target. The estimator in the article is combined from the unbiased MLE and the biased estimator of the covariance matrix of the Single Index Model. The main part of the article is to find the optimal shrinkage parameter, it done by using asymptotic theory.  
	
	\subsection{Shrinkage Target - Covariance of Single Index Model}	
	 Collection of assets is called a portfolio, each one of them has a return in some period of time. The single index model try to describe the components which affect on the return of assets in a portfolio. The idea of the model is that each return is effected from the total return of the portfolio. Basically it's a linear regression. For a specific period of time \(t\),  the return of an asset in time \(t\) is the target variable and the total return of the portfolio in time \(t\) is the explanatory variable.\\
	 
	\subsubsection{Single-index Model}
	Continue the notation from the introduction, \(x_{ij}\) is observation \(i\) of the return of asset \(j\). The number of the observations is \(n\) and the number of the parameters is \(p\). The vector \(X_i^T = [x_{i1}, x_{i2}, \dots, x_{ip}]\) is the observation \(i\) of \(p\) returns from the portfolio.\\
	 
	The model assumed that \(x_{ij}\) are generated by, 
	 
	\[x_{ij} = \alpha_{j} + \beta_{j}x_{i\cdot} + e_{ij}\]  
	\[\forall j = 1, 2, ..., p;  \forall i = 1, 2, ..., n\]
	\[Var(e_{ij}) := \delta_{jj}\]
	
	the variance of the residuals \(\delta_{jj}\) assumed to be uncorrelated with \(x_{ij}\) and to one another.\\
	
	Each return \(x_{ij}\) is regress on \(x_{i\cdot} = \sum_{j=1}^{p} x_{ij} \)  the market return of observation \(i\). The return of the market is computed for each observation \(i\), 
	\[x_{i\cdot} = [x_{1\cdot}, x_{2\cdot}, \dots, x_{n\cdot}]  = 
	\left[
	\sum_{j=1}^{p}x_{1j}, \sum_{j=1}^{p}x_{2j}, ..., \sum_{j=1}^{p}x_{nj} 
	\right] \]

	The model get \(n \times p\) observed matrix,
	\begin{align*} 
	X = 
	\begin{bmatrix}
	x_{11} & x_{21} & \dots  & x_{n1}  \\
	x_{12} & x_{22} & \dots  & x_{n2}  \\
	\vdots & \vdots & \dots  & \vdots  \\
	x_{1p} & x_{2p} & \dots  & x_{np}
	\end{bmatrix}
	\end{align*}
	
	
	and replace each entry by the linear equations, 
	\begin{align*} 
	\hat{X_{SIM}} =  
	\begin{bmatrix}
	\alpha_{1} + \beta_{1}x_{1\cdot} + e_{11} & \alpha_{1} + \beta_{1}x_{2\cdot} + e_{21} & \dots  & \alpha_{1} + \beta_{1}x_{n\cdot} + e_{n1}   \\
	\alpha_{2} + \beta_{2}x_{1\cdot} + e_{12} & \alpha_{2} + \beta_{2}x_{2\cdot} + e_{22} & \dots  & \alpha_{2} + \beta_{2}x_{n\cdot} + e_{n2}\\
	\vdots     & \vdots     & \ddots  & \vdots		\\
	\alpha_{p} + \beta_{p}x_{1\cdot} + e_{1p} & \alpha_{p} + \beta_{p}x_{2\cdot} + e_{2p}& \dots  & \alpha_{p} + \beta_{p}x_{n\cdot} + e_{np}
	\end{bmatrix}
	\end{align*}
	
	\paragraph{Variance of asset \(j\) in observation \(i\)}
	\[\phi_{jj} := Var(x_{ij})  = Var(\alpha_{j} + \beta_{j}x_{i\cdot} + e_{ij}) = Var(\beta_{j}x_{i\cdot} + e_{ij}) = Var(\beta_{j}x_{i\cdot}) + Var(e_{ij}) + 2Cov(\beta_{j}x_{i\cdot}, e_{ij}) \]
	\[= Var(\beta_{j}x_{i\cdot}) + Var(e_{ij}) = \beta_{j}^{2}Var(x_{i\cdot}) + \delta_{jj} =  \beta_{j}^{2}\sigma_{00}^{2} +\delta_{jj}  \]
	
	while \(\sigma_{00}^{2}\) is the variance of \(x_{i\cdot}\)\\
	
	\paragraph{Covariance of asset \(j\) and asset \(k\)}
	\[\phi_{jk} := Cov(x_{ij}, x_{ik})  = Cov(\alpha_{j} + \beta_{j}x_{i\cdot} + e_{ij}, \alpha_{k} + \beta_{k}x_{i\cdot} + e_{ik}) = Cov(\beta_{j}x_{i\cdot} + e_{ij}, \beta_{k}x_{i\cdot} + e_{ik}) \]
	
	\[= Cov(\beta_{j}x_{i\cdot}, \beta_{k}x_{i\cdot}) + Cov(\beta_{j}x_{i\cdot}, e_{ik}) + Cov(e_{ij}, \beta_{k}x_{i\cdot}) + Cov(e_{ij},e_{ik}) = Cov(\beta_{j}x_{i\cdot}, \beta_{k}x_{i\cdot})\]
	
	\[= \beta_{j}\beta_{k}Var(x_{i\cdot}) = \beta_{j}\beta_{k}\sigma_{00}^{2}\]
	
	It can be seen that the covariance is not depend on the observation.
	
	\paragraph{Matrix form}
	Writing together the variance of \(x_{ij}\) and the covariance of \(x_{ij}\) with \(x_{ik}\) in matrix form it obtain the covariance matrix of the model, 
	\[\Phi = \sigma_{00}^{2}\boldsymbol{\beta}\boldsymbol{\beta}' +\Delta \]
	
	while, \( \boldsymbol{\beta} = [\beta_{1}, \beta_{2}, ..., \beta_{i},.., \beta_{p}]^{T} \),  and \(\Delta\) is a diagonal matrix containing residual variances \(\delta_{jk} \)	
	
	\begin{align*} 
	\Phi =  
	\begin{bmatrix}
	\phi_{11}   & \phi_{12}  & \dots      & \phi_{1j}  & \dots       & \phi_{1p}  \\
	\phi_{21}   & \phi_{22}  & \dots      & \phi_{2j}  & \dots       & \phi_{2p}  \\
	\vdots      & \vdots     & \ddots     & \vdots	   & \ddots      & \vdots 	    \\
	\phi_{i1}   & \phi_{i2}  & \dots      & \phi_{ij}  & \dots       & \phi_{ip} 	\\
	\vdots      & \vdots     & \ddots     & \vdots	   & \ddots      & \vdots 	\\
	\phi_{p1}   & \phi_{p2}  & \dots      & \phi_{pj}  & \dots       & \phi_{pp}
	\end{bmatrix}
	+\Delta
	=
	{Var(x_{i\cdot})}
	\begin{bmatrix}
	\beta_1\beta_1 & \beta_1\beta_2 & \dots  & \beta_1\beta_j & \dots  & \beta_1\beta_p \\
	\beta_2\beta_1 & \beta_2\beta_2 & \dots  & \beta_2\beta_j & \dots  & \beta_2\beta_p \\
	\vdots         & \vdots         & \ddots & \vdots	      & \ddots & \vdots 	    \\
	\beta_j\beta_1 & \beta_j\beta_2 & \dots  & \beta_j\beta_j & \dots  & \beta_j\beta_p \\
	\vdots         & \vdots         & \ddots & \vdots	      & \ddots & \vdots 	    \\
	\beta_p\beta_1 & \beta_p\beta_2 & \dots  & \beta_p\beta_j & \dots  & \beta_p\beta_p \\
	\end{bmatrix}
	+\Delta
	\end{align*}
	\begin{align*}
	=
	\frac{1}{Var(x_{i\cdot})}
	\begin{bmatrix}
	Cov(x_{i\cdot}, x_{i1})Cov(x_{i\cdot}, x_{i1}) & \dots      	  
	& Cov(x_{i\cdot}, x_{i1})Cov(x_{i\cdot}, x_{ij}) & \dots       
	& Cov(x_{i\cdot}, x_{i1})Cov(x_{i\cdot}, x_{ip}) 
	\\
	Cov(x_{i\cdot}, x_{i2})Cov(x_{i\cdot}, x_{i1}) & \dots      	  
	& Cov(x_{i\cdot}, x_{i2})Cov(x_{i\cdot}, x_{ij}) & \dots       
	& Cov(x_{i\cdot}, x_{i2})Cov(x_{i\cdot}, x_{ip}) \\		
	\vdots  & \ddots  & \vdots	& \ddots  & \vdots 	 \\
	Cov(x_{i\cdot}, x_{ij})Cov(x_{i\cdot}, x_{i1})  & \dots      	  
	& Cov(x_{i\cdot}, x_{ij})Cov(x_{i\cdot}, x_{ij})  & \dots       
	& Cov(x_{i\cdot}, x_{ij})Cov(x_{i\cdot}, x_{ip}) 
	\\		
	\vdots  & \ddots  & \vdots	& \ddots  & \vdots 	\\
	Cov(x_{i\cdot}, x_{ip})Cov(x_{i\cdot}, x_{i1}) & \dots      	  
	& Cov(x_{i\cdot}, x_{ip})Cov(x_{i\cdot}, x_{ij}) & \dots       
	& Cov(x_{i\cdot}, x_{ip})Cov(x_{i\cdot}, x_{ip}) 
	\\
	\end{bmatrix}
	+\Delta
	\end{align*}
	
	while \(\Delta\) is,
	\begin{align*}
	\Delta = 
	\begin{bmatrix}
	\delta_{11} & 0 & \dots  & 0 & \dots  & 0 \\
	0 & \delta_{22} & \dots  & 0 & \dots  & 0 \\
	\vdots         & \vdots         & \ddots & \vdots	      & \ddots & \vdots 	    \\
	0 & 0 & \dots  & \delta_{kk} & \dots  & 0 \\
	\vdots         & \vdots         & \ddots & \vdots	      & \ddots & \vdots 	    \\
	0 & 0 & \dots  & 0 & \dots  & \delta_{pp} \\
	\end{bmatrix}
	\end{align*}
	
	

	
	\subsubsection{Fitting the Model}
	The model is fitted by running a simple linear regression for each return \(j\) separately. In order to understand how the process is done, let's look on a specific return \(j=1\).\\
	
	Each return has \(n\) samples \(x_{1,1}, x_{2,1}, ..., x_{n,1}\) following linear equation of the model for each samples is,
	\[x_{i1} = \alpha_{1} +\beta_{1}x_{i\cdot} + e_{1j} \  \  \ \forall i =1,2,\dots, n \]
	
	the target variable is \(x_{i1}\), the explanatory variable is \(x_{i\cdot}\) the slope of the regression is \(\beta_1\), the constant of the regression is \(\alpha_1\).\\
	
	Since there is \(n\) samples, there is \(n\) realizations of the linear equation. Similarly to simple linear regression the slope can be estimated by, 
	\[\beta_1 = \frac{Cov(x_{i\cdot}, x_{i1})}{Var(x_{i\cdot})} \] 

	the same fitting procedure that done for \(j=1\) is done for each return \(j=1,2,\dots, p\), the vector that contain all the slopes of the regression is \(\boldsymbol{\beta}\).
	 
	\[\boldsymbol{\beta} = 
	\begin{bmatrix}
	\beta_{1} \\
	\beta_{2} \\
	\vdots \\
	\beta_{p} \\
	\end{bmatrix}
	=
	\begin{bmatrix}
	\frac{ Cov(x_{i\cdot}, x_{i1}) }{Var(x_{i\cdot})}\\
	\frac{ Cov(x_{i\cdot}, x_{i2}) }{Var(x_{i\cdot})} \\
	\vdots \\
	\frac{ Cov(x_{i\cdot}, x_{ip}) }{Var(x_{i\cdot})} \\
	\end{bmatrix}
	 \]
	 
	The estimation of \(\boldsymbol{\beta}\) is denote by \( \boldsymbol{b} = (b_1, b_2, .., b_n)\). The estimation of \(\Phi = \sigma_{00}^{2}\boldsymbol{\beta \beta'} +\Delta \) is denote by \(\boldsymbol{F} = s_{00}^{2}bb' + D \) where \(s_{00}^{2}\) is the estimated variance of \(x_{i\cdot}\) and \(D\) is a diagonal matrix with entries of the variance of the residuals.\\  
	

	
	%\(\beta_{j}\) is capturing the contribution of asset \(j\) to the volatility of single factor.\\
	\subsubsection{Covariance matrix of the single index model}
	Focusing on specific entry, may help to understand the nature of \(\Phi\).
	
	\[\phi_{21} = \frac{1}{Var(x_{i\cdot})}Cov(x_{i\cdot}, x_{i2})Cov(x_{i\cdot},x_{i1})\]
	
	\(x_{i\cdot} = [x_{1\cdot}, x_{2\cdot}, \dots , x_{n\cdot}]\) - each element is sum of all the returns in different observation (time). 
	
	\(x_{i2} = [x_{12}, x_{22}, \dots , x_{n2}] \) - each element of the vector is return of asset 2 in different observation (time) 
	
	\(x_{i1} = [x_{11}, x_{21}, \dots , x_{n1}] \) - each element of the vector is return of asset 1 in different observation (time)\\
		
	\(Cov(x_{i\cdot}, x_{i2})\) -  covariance of asset 2 and the total returns.\\
	\(Cov(x_{i\cdot}, x_{i1})\) -  covariance of asset 1 and the total returns. \\
	\(Var(x_{i\cdot}) \) - variance of the total returns.\\
	
	Now, it possible to see what \(\phi_{21} \) is capturing. The numerator is the covariance of asset 1 with total returns multiply by the covariance of asset 2 with total returns. The denominator is the variance of total returns. \(\phi_{21}\) is capturing the share of asset 1 and asset 2 variances from the total variance of the returns. 
	
	\paragraph{Number Of Parameters To Estimate}
	For each asset, there is one parameter to estimate and one parameter of the index, total of \(\ p + 1 \) parameters to estimate. Comparing to the MLE with \(\frac{(p+1)}{2}p\) parameters to estimate, the SIM has less parameters to estimate, therefore the variance expected to be lower from the variance of \(S\).
	
	
	\subsubsection{Shrinkage Estimator}
	The suggested improved estimator is done by shrinkage method. In order to perform this method, two estimators is needed. The first is the MLE unbiased estimator \(S\). The second estimator is the shrinkage target, it will be the covariance matrix of the single index model \(F\). The weight between the estimators is,  \(0 < \alpha < 1\). 
	\[
	\alpha
	\boldsymbol{F} +
	\left(
	1- \alpha  
	\right)
	\boldsymbol{S}
	\] 
	
	\subsubsection{The Optimal Shrinkage Intensity}
	Choosing the optimal shrinkage parameter is done by minimizing the risk function.\\

	The loss function is define with Forbenius norm,
	\[L(\alpha) = 
	\lVert 
	\alpha\boldsymbol{F} + \left(1- \alpha  \right) \boldsymbol{S}
	-\Sigma 
	\lVert^{2}_{f} \]
	
	The risk function, 
	
	\[R(\alpha) = E(L(\alpha)) =
	E\left[
	\lVert 
	\alpha\boldsymbol{F} + \left(1- \alpha  \right) \boldsymbol{S}
	-\Sigma 
	\lVert^{2}_{f}   
	\right]
	\]
	\[
	= 
	E\left[   
	\sum_{i=1}^{p}\sum_{j=1}^{p}
	\left(
	\alpha f_{ij}  + (1-\alpha)s_{ij} -\sigma_{ij} 
	\right)^{2}
	\right]
	=
	\sum_{i=1}^{p}\sum_{j=1}^{p}E[	\left(
	\alpha f_{ij}  + (1-\alpha)s_{ij} -\sigma_{ij} 
	\right)^{2}]
	\]
	
	\[
	=
	\sum_{i=1}^{p}\sum_{j=1}^{p}
	Var(\alpha f_{ij} + (1-\alpha)s_{ij} - \sigma_{ij}) + \left(E\left[\alpha f_{ij} + (1-\alpha)s_{ij} - \sigma_{ij} \right] \right)^{2}
	\]
	
	 
	\[=
	\sum_{i=1}^{p}\sum_{j=1}^{p}
	Var(\alpha f_{ij} + (1-\alpha)s_{ij}) 
	+\left[\alpha E(f_{ij}) + E(s_{ij}) -\alpha E(s_{ij}) - E(\sigma_{ij})
	\right]^{2} 
	\]
	
	\[=
	\sum_{i=1}^{p}\sum_{j=1}^{p}
	Var(\alpha f_{ij}) + Var((1-\alpha)s_{ij})) +2Cov(\alpha f_{ij}, (1-\alpha)s_{ij})) 
	+\left[\alpha \phi_{ij} + \sigma_{ij} -\alpha \sigma_{ij} - \sigma_{ij}
	\right]^{2} 
	\]
	
	\[=
	\sum_{i=1}^{p}\sum_{j=1}^{p}
	\alpha^{2}Var(f_{ij})+(1-\alpha)^{2}Var(s_{ij})+2\alpha(1-\alpha)Cov(f_{ij},s_{ij}) +\alpha^{2}(\phi_{ij}-\sigma_{ij})^{2} 
	\]
	
	first derivation,
	\[
	R'(\alpha) = 2
	\sum_{i=1}^{p}\sum_{j=1}^{p}
	\alpha Var(f_{ij}) - (1-\alpha)Var(s_{ij}) +(1-2\alpha)Cov(f_{ij},s_{ij}) +\alpha(\phi_{ij}-\sigma_{ij})^{2}
	\]
	
	\[=
	2\sum_{i=1}^{p}\sum_{j=1}^{p}
	\alpha \left[ Var(f_{ij}) + Var(s_{ij}) -2Cov(f_{ij},s_{ij}) + (\phi_{ij}-\sigma_{ij})^{2}\right]
	- Var(s_{ij}) + Cov(f_{ij},s_{ij})
	\]
	
	setting \(R'(\alpha) \) to zero, the \(\alpha \) that bring \(R(\alpha) \) to minimum is,
	
	\[
	\alpha^* = 
	\frac{\sum_{i=1}^{p}\sum_{j=1}^{p}Var(s_{ij})-Cov(f_{ij},s_{ij})}
	{\sum_{i=1}^{p}\sum_{j=1}^{p}Var(f_{ij}-s_{ij}) + (\phi_{ij}-\sigma_{ij})^{2}}
	\]
	
	second derivation,
		\[
	R''(\alpha) = 2
	\sum_{i=1}^{p}\sum_{j=1}^{p}
	Var(f_{ij}) + Var(s_{ij}) - 2Cov(f_{ij},s_{ij}) + (\phi_{ij}-\sigma_{ij})^{2}
	\]
	
	\(R''(\alpha) > 0 \), therefore it indeed a minimum point.
	
	
	\subsubsection{Shrinkage Parameter}
	The next step is using asymptotic theory and finding the \(\alpha*\) without unobserved components.\\
	\[
	n\alpha^* = 
	\frac{\sum_{i=1}^{p}\sum_{j=1}^{p}Var(\sqrt{n} s_{ij})-Cov(\sqrt{n}f_{ij},\sqrt{n}s_{ij})}
	{\sum_{i=1}^{p}\sum_{j=1}^{p}Var(f_{ij}-s_{ij}) + (\phi_{ij}-\sigma_{ij})^{2}}
	\]
	
	\paragraph{Theorem 1}
	Let the following denotation, 
	\[\pi =  \sum_{i=1}^{p} \sum_{j=1}^{p} \lim_{n\to\infty} \sum_{n=1}^{n} Var(\sqrt{n}s_{ij}) \]
	while, \(s_{ij} = \frac{1}{N} \sum_{n=1}^{N} x_{in}x_{jn}\)
	\[\rho = \sum_{i=1}^{p} \sum_{j=1}^{p} \lim_{N\to\infty} \sum_{n=1}^{N} Cov(\sqrt{N}f_{ij}, \sqrt{N}s_{ij} )   \]
	\[\gamma =  \sum_{i=1}^{p} \sum_{j=1}^{p} \left( \phi_{ij} -\sigma_{ij}\right)^{2} \]
	
	
	then the optimal shrinkage \(\alpha^{*}\) satisfies, 
	\[\alpha^{*} = \frac{1}{N} \frac{\pi - \rho}{\gamma} + O \left( \frac{1}{N^{2}} \right) \]
	
	\paragraph{Lemma 1} A consist estimator for \(\pi \)
	\[\hat{\pi} = \sum_{i=1}^{p} \sum_{j=1}^{p} \hat{\pi}_{ij} \]
	\[\hat{\pi}_{ij} = \frac{1}{N} \sum_{n=1}^{N} \left( x_{in}x_{jn} - s_{ij} \right)^{2}  \]
	
	\paragraph{Lemma 2} A consist estimator for \(\rho\)
	\[\hat{\rho} = \sum_{i=1}^{p} \sum_{j=1}^{p} \hat{\rho}_{ij} \]
	
	for \(i=i\), 
	\[\hat{\rho}_{ii} = \hat{\pi}_{ii} \]
	
	for \(i \neq j \), 
	\[ \hat{\rho}_{ij} = \frac{1}{N} \sum_{n=1}^{N} \hat{\rho}_{ijn} \]
	
	\[\hat{\rho}_{ijn} = \frac{s_{j\cdot}s_{\cdot \cdot}x_{in} + s_{i\cdot}s_{\cdot\cdot}x_{jn} - s_{i\cdot} s_{j\cdot}   \left(x_{\cdot n}-x_{\cdot\cdot}\right) }{s_{\cdot\cdot}^{2} } \left(x_{\cdot n}-x_{\cdot\cdot}\right)x_{in}x_{jn} - f_{ij}s_{ij}\]
	
	while,
	\[s_{\cdot\cdot}^{2} = var(x_{i\cdot}) \]
	\[s_{k\cdot} = cov(x_{i\cdot}, x_{ik}) \]
	
	
	\paragraph{Lemma 3} A consist estimator for \(\gamma_{ij} \)
	\[\hat{\gamma} =  \sum_{i=1}^{p} \sum_{j=1}^{p} \left( f_{ij} - s_{ij}\right)^{2} \]
	
	\subsubsection{The Estimator}
	Combine theorem 1, lemmas 1 2 and 3 : 
	\[
	\frac{k}{N}\boldsymbol{F} +
	\left(1- \frac{k}{N}  \right)
	\boldsymbol{S}
	\]
	
	where, \(k = \frac{\hat{\pi} - \hat{\rho} }{ \hat{\gamma} }\) is called the constant shrinkage parameter.
	
	\subsection{Simulations}
	Purpose of the shrinkage method is to gain better result from MLE the usual estimator. Testing the estimators require realizations of the random vector \(X\),  population covariance matrix of \(X\) and evaluation system.   
	
	\subsubsection{Evaluation Of The Estimator}
	Given a true covariance matrix and an estimator \(\Sigma , \hat{\Sigma}\) the following can be computed,\\
	
	for specific entry,
	\[\hat{\bar{\Sigma}}_{ij} = \frac{1}{L}\sum_{l=1}^{L}\hat{\Sigma}_{ij} \]
	where \(L\) is the number of repetitions.
	
	\[ Bias^2( \hat{\Sigma}_{ij} ) = \frac{1}{L}\sum_{l=1}^{L} \left( \hat{{\Sigma}}_{ij}^{l} - \Sigma_{ij}
	\right)^{2} \]
	
	\[Var(\hat{\Sigma}_{ij}) = \sum_{l=1}^{L} 
	\left(
	 \hat{\Sigma}_{ij}^{l} -\hat{\bar{\Sigma}}_{ij} 
	 \right)^{2}
	  \]

	\[
	EMSE(\hat{\Sigma}_{ij}) = Bias^2( \hat{\Sigma}_{ij} ) + Var(\hat{\Sigma}_{ij})
	\]
	for the entire matrix, 
	\[Bias^2( \hat{\Sigma}) = \sum_{i=1}^{p} \sum_{j=1}^{p} Bias^2( \hat{\Sigma}_{ij} )
	\]
	
	\[Var( \hat{\Sigma}) = \sum_{i=1}^{p} \sum_{j=1}^{p} Var(\hat{\Sigma}_{ij})
	\]
	
	\[
	EMSE(\hat{\Sigma}) = Bias^2( \hat{\Sigma} ) + Var(\hat{\Sigma})
	\]
	
	\subsubsection{Data}
	
	\paragraph{Market Model}
	Generating data according to simple market model, 
	\[r_{i,t} = \beta_{i} \cdot r_{M, t} + \epsilon_{i,j} \]
	\[ \beta_{1} +  \beta_{2} + ... +  \beta_{p}  = 1\]
	where \(r_{i,t}\) is the return of asset \(i\) on time \(t\). \(r_{M,t}\) is the total return of the market in time \(t\). \(\epsilon_{i,j}\) is normal random variable. 
	
	\paragraph{Multi-Normal distribution} Generating data with known covariance matrix which will take the role of the true covariance matrix in the simulations. 
	
	\subsubsection{Results}
	Simulations details:\\
	\begin{tabular}{|c|c|c|}
		\hline 
		                               & \textbf{S\&P 500} & \textbf{Market Model} \\ 
		\hline 
		number of simulations          & 10                & 100 \\ 
		\hline 
		number of variables - \(p\)    & 491               & 100 \\
		\hline
		number of observations - \(N\) & 250               &  100 \\ 
		\hline 
		sample size                    & 0.8               & 0.6 \\
		\hline 
		\(\frac{p}{n}  \)              & 2.455             &  1.66 \\ 
		\hline 
		\(\pi\) - error on the sample covariance matrix                     & 2.6316 & 63.4444 \\
		\hline
		\(\rho\) - covariance between the estimation error of MLE and SIM   & 2.5782 & 62.2252 \\
		\hline
		\(\gamma\) - misspecification of the single-index model             & 0.0775 & 1.0280  \\
		\hline
		shrinkage constant \(k\)                                            & 0.6126 & 1.162   \\
		\hline
		shrinkage parameter    \(\alpha\)                                   & 0.0030 & 0.01937 \\
		\hline
	\end{tabular}\\
	\\
	\\
	Evaluation of the estimators:\\
	\begin{tabular}{|c|c|c|c|c|}
	\hline 
	        & \multicolumn{2}{c|}{\textbf{S\&P 500}}              & \multicolumn{2}{c|}{\textbf{Market Model}} \\ 
	\hline 
                               & MLE    & shrinkage               & MLE    & shrinkage \\
	\hline 
	\(Bias^2( \hat{\Sigma})\)  & 0.0533 & 0.0477                  & 0.6759 &  0.6583   \\ 
	\hline 
	\(Var( \hat{\Sigma})\)     & 0.0029 & \(3.4607\cdot 10^{-7}\) & 0.4023 &  0.0001   \\ 
	\hline
	\(EMSE(\hat{\Sigma})\)     & 0.05637 & 0.0477                  & 1.0700 &  0.6584    \\ 
	\hline 
	\end{tabular}\\
	\\
	\\
	Variance the shirnkage target vs  variance of MLE:\\
	\begin{tabular}{|c|c|c|}
		\hline 
		& \textbf{S\&P 500} & \textbf{Market Model} \\ 
		\hline 
		variance of MLE & 0.0028 &  0.4151 \\ 
		\hline 
		variance of shrinkage target & \(3.2811 \cdot 10^{-7}\) &  \(6.9581 \cdot 10^{-5}\) \\ 
		\hline 
	\end{tabular}\\
	

	\paragraph{Known covariance with multi-normal distribution}.\\
	The result were not good enough. For \(\frac{p}{n} > 1 \) the shrinkage parameter was above 1.   
	
	\subsection{Discussion}
	The shrinkage constant converge to zero as the number of observations goes up. \\
	The method has good performance only when the eigenvectors are closed.

	
	 
	
	\newpage
	\printbibliography
	

\end{document}














